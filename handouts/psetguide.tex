\documentclass{article}
\usepackage{fullpage}
\usepackage{amsmath,amsthm,amssymb,amsfonts}
\usepackage{graphicx,epstopdf,fancyhdr,xspace,algorithm,algorithmic}
\usepackage[space]{grffile}
\usepackage{epsfig}
\usepackage[space]{grffile}
\usepackage{titlesec}
\usepackage[table]{xcolor}
\usepackage[hyphens]{url}
\usepackage{enumitem} % package used to generate bulleted lists 
\usepackage{tikz}
\usetikzlibrary{arrows,automata}
\theoremstyle{definition}
\renewcommand{\epsilon}{\varepsilon}
\newtheorem{problem}{Problem} %
\cfoot{\thepage}
\renewcommand{\headrulewidth}{0.4pt}
\renewcommand{\headwidth}{\textwidth}
%\renewcommand{\footwidth}{\textwidth}
%\addtolength{\headwidth}{\marginparwidth}
%\addtolength{\headwidth}{\marginparsep}
\renewcommand{\footrulewidth}{0.4pt}
\usepackage{multicol}

\newcommand{\mymin}{\ensuremath{\mbox{\sc FindMin}}\xspace} 
\newcommand{\ms}{\ensuremath{\mbox{\sc Mergesort}}\xspace} 
\newcommand{\qs}{\ensuremath{\mbox{\sc Quicksort}}\xspace} 
\titleformat*{\section}{\Large\bfseries}
\titleformat*{\section}{\large\bfseries}
\usepackage{hyperref}
\hypersetup{
    colorlinks=true,
    linkcolor=blue,
    filecolor=magenta,      
    urlcolor=cyan,
}
 
\urlstyle{same}
\PassOptionsToPackage{hyphens}{url}
\definecolor{grayX}{gray}{0.95}
\begin{document}
\pagenumbering{gobble}
%\homework{Problem Set Advice}{Shikha Singh}{}{}{}
%\section*{CS256: Problem Set Advice}

\begin{center}
    {\Large \sc CS256 Problem Set Advice}\\ 
\end{center}

\begin{center}
{\em Instructor:} Shikha Singh\\
\end{center}

\section{General Advice}

\paragraph{Start early.}
Algorithm design takes time, and even simple algorithms
can be surprisingly tricky to develop. I recommend reading over the problems as soon as
the problem set is released so that you have the time to play around with them over the
course of the week.  


In my experience, sometimes just having the problem in the back of your head for a while helps, even
if you aren't actively working on it.

\paragraph{Collaborate, seek help but work on your own first.}  
  Although you are encouraged to collaborate, work in groups, and seek help from TAs, I strongly recommend not doing so until you have tried each of the problems and given them some independent thought. 

The homework problems tend to have solutions that are not particularly complicated but which require some
insight to discover. If you immediately start working on the assignments in a group, you
will miss out on the opportunity to come up with these insights on your own. 

\paragraph{Don't submit your first draft.} 
When you come up with an algorithm and its analysis, your first iteration will likely have some rough edges or unnecessary parts. Taking the time to clean up your proofs and clarify your algorithm will
both cement your understanding of the material and help your overall assignment grade.  You will also notice gaps in logic, or bugs that you did not the first time. 

In contrast to programming assignments, unfortuntately there is no autochecker for proofs---so you have to `debug' your proofs by rereading them yourself.

\section{Algorithms and Proofs}

\paragraph{Algorithm write up.}
 In this course, you will often be asked to {\em give, develop, design or describe} an algorithm for a particular problem.  The expectation is that you give a clear concise and complete description in prose of how your algorithm works.  As you write up an algorithm, you need to present enough detail so that the reader can accurately analyze the algorithm's correctness and runtime, but not so much detail that the high-level idea isn't
clear. 
Low-level detailed pseudocode is often hard to understand (and is insufficient on its own). Pseudocode should only be used if it is absolutely crucial to your analysis.  

\paragraph{Running time and correctness.} If you are asked to design an algorithm, unless otherwise stated, you must also analyze its running time and correctness. Often the problem will require that your algorithm be of a certain time complexity, e.g. linear time, in which case you must argue why it meets those requirements. 

\begin{itemize}[leftmargin = 10pt, label = $\bullet$]
    \item {\bf Running time.} For running time analysis, you do not need to invoke formal first-principles of how $O$
and $\Omega$ are defined but rather a more high-level argument that accounts for the various operations of your algorithm.  As the course proceeds, we'll learn several techniques to analyze running time.
 \item {\bf Space.} While it is good practice to include an analysis of the space complexity of the algorithm (and we will do so in class), you are not required to include it unless the question asks for it explicitly.
 \item {\bf Correctness.} When proving that your algorithm works correctly, you must give a rigorous mathematical proof.  It is always a good idea to also include the general idea of why your algorithm works but that does not make up for a proof.  We will learn various techniques for proving correctness of different types of algorithms throughout the course---and you already have learned some (e.g. induction) in CS136. 
\end{itemize}

\section{Grading}

Solutions to assignment questions should be written cleanly and concisely.  Writing good proofs is both about having the correct logic, and effectively expressing that logic to the reader.

Your solutions will be graded on their correctness and clarity.  A representative rubric for a 10-point proof-based question is presented below.

\smallskip
\begin{tabular}{rl}
  \textbf{10:}& 	The solution is clear and correct.\\
\textbf{9:} &	The solution is clear but contains a few mistakes, but they are minor or of little significance.\\
\textbf{8:} &	The solution hits on the main points, but has at least one logical gap.\\
\textbf{7:} &	The solution is significantly unclear or contains major gaps, but parts of it are salvageable.\\
\textbf{6:} &	The solution is just plain wrong or so unclear it cannot be followed.\\
\textbf{0:} &	No attempt is made at solving the problem.
\end{tabular}
\smallskip

\section{Latex Typesetting Requirements}

Each assignment will also have a (small but noticeable) number of points dedicated to Latex typesetting.  This is to encourage good habits and correct usage.  The following is a list of common mistakes to bear in mind while typesetting latex---it's not meant to be exhaustive.

\begin{itemize}[noitemsep]
  \item All variables and equations should be in math mode---one should write $O(n)$ rather than O(n), and $n < m$ rather than n < m.  Both inline math mode (using \texttt{\$\ldots \$}) and display math mode (using \texttt{\textbackslash[ \ldots \textbackslash]}) are acceptable.
  \item Whitespace and indentation should be done with correct latex usage.  The command \texttt{\textbackslash\textbackslash} ~should only be used to force a line break when necessary, not to end a paragraph (when a blank line would do).
  \item Text should always fit on the page, as otherwise it is impossible to read.  \mbox{Here is an example of text not fitting on the page.}
  \item Math mode should not be used except to typeset math.  To italicize text, use \verb|\textit{}|.
  \item Environments should be used correctly--in particular, solutions should be within the designated solution environment.
\end{itemize}

One quick sanity check to see if the above requirements are being followed is to check for latex errors during compilation---oftentimes, a latex error indicates that you are doing something wrong.




\section*{Acknowledgement}
This handout is based on similar handouts by Tim Roughgarden and Keith Schwarz.  The rubric is adapted from Tom Murtagh and the LaTeX guide is from Sam McCauley.

\end{document}
